\documentclass[conference]{IEEEtran}
\IEEEoverridecommandlockouts
% The preceding line is only needed to identify funding in the first footnote. If that is unneeded, please comment it out.
%Template version as of 6/27/2024

\usepackage{cite}
\usepackage{amsmath,amssymb,amsfonts}

\usepackage{graphicx}
\usepackage{textcomp}
\usepackage{xcolor}
\def\BibTeX{{\rm B\kern-.05em{\sc i\kern-.025em b}\kern-.08em
    T\kern-.1667em\lower.7ex\hbox{E}\kern-.125emX}}


\usepackage{subcaption}

% algorithm psuedocode
\usepackage{algorithm} 
\usepackage{algpseudocode} 

\usepackage{hyperref}


\begin{document}

\title{RBE 502 Catapult Project Report}

\author{\IEEEauthorblockN{Gregory Kyle Jones}
%\IEEEauthorblockA{}
}

\maketitle


\section{Introduction}
Controlling a robot arm to throw an object is a challenging task. The goal of
this work is to design at least one controller to follow trajectories that could
be used to throw an object with a UR5 robot arm. The task of picking up an
object was excluded in order to focus more on the controller design.

\section{Methods}

\subsection{Simulation Environment and Tools}
Several tools were used to simulate the robot and its environment, plan the
motion, and produce results. Gazebo was used for simulation, MoveIt was used for
motion planning, ROS2 was used for robot communication and control, and Rviz was
used for visual debugging. For simulation, a URDF model of a UR5 robot arm was
used, however a gripper had to be attached and MoveIt configured
accordingly. This simulation setup was used as part of the design process such
that simulation in Matlab was not necessary.

Docker was used to containerize all of these tools, which helped track
package dependencies. Docker proved to be particularly helpful for making quick
changes to the version of the operating system and software tools. Finally, all
code changes were tracked in a github repository.

\subsection{Projectile Trajectory Planning}
Planning of the projectile trajectory was done theoretically, and is included
here for completeness, but did not end up in the final implementation for
simplicity. The projectile trajectory calculation was represented as a
two dimensional problem in which the motion occurs in an $x-y$ plane. The inputs
to this problem include the release position $(x_0, y_0)$, the final position of
the target $(x_f, y_f)$, the release angle $\theta$, and gravity $g$. Using
standard projectile motion equations from~\cite{william25} and rearranging to
solve for the release velocity $v$ gives the equation
\begin{equation*}
  v = \sqrt{ \left(\frac{g(x_f-x_0)^2}{2 cos^2\theta ((x_f-x_0) tan \theta - y_f + y_0)} \right ) }
\end{equation*}

\subsection{Motion Planning}
With the object assumed to already be grasped, the first trajectory move the
robot to a start configuration for the throw motion, and the second trajectory
performed the throw motion. If the trajectory planning for the throwing motion
was done in joint space, then the end effector would likely be rotating close to
the point of releasing the projectile object. Any small error in the robot
timing to release the object would cause a large error in reaching the target.

To reduce this error amplification, the throwing motion instead used a straight
line end effector path/trajectory in task space, as well as a constant task
space target velocity for some duration leading up to the release point. The
timestamps of each sample point in this linear trajectory could then be modified
based on the linear segments with parabolic blends (LSPB) method
\cite{mark2020control}. With the distance of the end effector along this linear
path denoted as $s(t)$ where $t$ is time. There is a constant acceleration from
$t_0$ to $t_b$, then a constant velocity from $t_b$ to $t_f-t_b$, and finally a
constant decceleration from $t_f-t_b$ to $t_f$. The desired motion is described
with the following equations.
\begin{equation*}
  s(t) =
  \begin{cases}
    s_0+\frac{\alpha}{2}t^2, & 0 \leq t < t_b \\
    \frac{s_f+s_0-Vt_f}{2} + Vt & t_b \leq t < t_f-t_b \\
    s_f-\alpha {t_f}^2+\alpha t_f t-\frac{\alpha}{2} t^2 & t_f-t_b \leq t_b \leq t_f
  \end{cases}
\end{equation*}

\begin{equation*}
  \dot{s}(t) =
  \begin{cases}
    \alpha t \\
    V \\
    \alpha t_f - \alpha t
  \end{cases}
\end{equation*}

\begin{equation*}
  \ddot{s}(t) =
  \begin{cases}
    \alpha \\
    0 \\
    -\alpha
  \end{cases}
\end{equation*}

\begin{equation*}
  \alpha = \frac{V}{t_b}
\end{equation*}


\subsection{Dynamic Model}
The dynamic model of the selected 6 DoF robot arm (Universal Robots UR5) was
derived analytically using Matlab and equations based on Lagrange's
method. Lagrange's method applied to an $n$ DoF manipulator generates the
equations of motion using
\begin{equation*}
  \frac{d}{dt} \left[ \frac{\partial \mathcal{L}(\boldsymbol{q}, \dot{\boldsymbol{q}})}{\partial \dot{\boldsymbol{q}}} \right] - \frac{\partial \mathcal{L}(\boldsymbol{q}, \dot{\boldsymbol{q}})}{\partial \dot{\boldsymbol{q}}} = \boldsymbol{\tau}
\end{equation*}

where $\boldsymbol{q}=[q_1, q_2, ..., q_n]^T$ is the generalized joint
coordinate vector, $\boldsymbol{\tau}$ is the generalized joint force vector,
and $\mathcal{L}(\boldsymbol{q}, \dot{\boldsymbol{q}})$ is the Lagrangian. The
Lagrangian is given by
\begin{equation*}
  \mathcal{L}(\boldsymbol{q}, \dot{\boldsymbol{q}}) = \mathcal{K}(\boldsymbol{q}, \dot{\boldsymbol{q}}) - \mathcal{U}(\boldsymbol{q})
\end{equation*}

where $\mathcal{K}$ is the total kinetic energy, and $\mathcal{U}$ is the total
potential energy due to conservative forces such as gravity and energy stored in
compressed springs.

The final equations of motion can then be represented in compact form as
\begin{equation*}
  M(\boldsymbol{q})\ddot{\boldsymbol{q}} + C(\boldsymbol{q}, \dot{\boldsymbol{q}})\dot{\boldsymbol{q}} + \boldsymbol{g}(\boldsymbol{q}) = \boldsymbol{\tau}
\end{equation*}

where $M(\boldsymbol{q})\ddot{\boldsymbol{q}}$ is the inertia matrix,
$C(\boldsymbol{q}, \dot{\boldsymbol{q}})$ is the coriolis matrix, and
$\boldsymbol{g}(\boldsymbol{q})$ is the gravity vector.

During implementation is was found that the matrices $M$ and $C$ were large and
complex terms. Some approximations were made to reduce the complexity of these
terms. These included setting several center of mass dimensions to zero, as well
as using point masses only for ineria terms (no inertia tensor). The
coriolis term still remained the most complex so it was neglected, assuming the
inertia matrix would contribute more to the torques.

\subsection{Controllers}
Only controllers that satisfy the motion control objective were selected for
implementation. That is they produce $\boldsymbol{\tau}$ such that the joint
positions $\boldsymbol{q}$ follow the time changing desired joint positions
$\boldsymbol{q_d}$ accurately. The selected controllers include computed torque
and PD plus feedforward~\cite{kelly2007control}. Both of these controllers take
into account the full dynamic model.

\subsubsection{Computed Torque}
The controller for computed torque uses the robot dynamics in the feedback loop
and is given as
\begin{equation*}
  \boldsymbol{\tau} = M(\boldsymbol{q}) \left [ \ddot{\boldsymbol{q}}_d + K_v \dot{\tilde{\boldsymbol{q}}} + K_p \tilde{\boldsymbol{q}} \right ] + C(\boldsymbol{q}, \dot{\boldsymbol{q}})\dot{\boldsymbol{q}} + \boldsymbol{g}(\boldsymbol{q})
\end{equation*}

where $K_v$ and $K_p$ are symmetric positive definite design matrices and
$\tilde{\boldsymbol{q}} = \boldsymbol{q}_d - \boldsymbol{q}$ is the position
error. A diagram of the computed torque controller is shown in
Fig. \ref{fig:comp-torque}.

\begin{figure}[!t]
  \includegraphics[width=0.5\textwidth]{controller-comp-torque.png}
  \caption{Computed Torque Control Diagram}
  \label{fig:comp-torque}
\end{figure}

These design matrices can be selected to be diagonl, which cause the
dynamics of errors of the closed loop system to become decoupled, such that the
behavior of each joint position error is governed by a second order linear
differential equation. These matrices can then be selected as
\begin{align*}
  K_p &= diag\{\omega_1^2, ... , \omega_n^2\} \\
  K_v &= diag\{2\omega_1, ... , 2\omega_n\}
\end{align*}

which causes each joint to respond as a critically damped linear system with
bandwith $\omega_i$. Bandwidth $\omega_i$ defines the allowable range for joint
velocity $i$ before more than a $90$ degree phase lag occurs, as well as the
decay rate of the error for joint $i$.

During implementation it was found that the approximations made to the dynamics
resulted in no torque generated for the final joint. More specifically, the
center of mass of the final link was assumed to be located at the center of the
final joint, and the rotational inertia was assumed to be zero. Due to errors in
these assumptions, the joint would have a large swaying motion instead of
remaining fixed, based on a fixed desired position. To address this, a separate
PD controller was designed for the sixth joint only.

\subsubsection{PD Plus Feedforward}
The controller for PD plus feedforward uses the dynamics in a feedforward manner
by using the position and velocity of the desired trajectory. It also includes a
PD feedback controller. The complete controller is given by
\begin{equation*}
  \boldsymbol{\tau} = K_p \tilde{\boldsymbol{q}} + K_v \dot{\tilde{\boldsymbol{q}}} + M(\boldsymbol{q}_d)\ddot{\boldsymbol{q}}_d +  C(\boldsymbol{q}_d, \dot{\boldsymbol{q}}_d)\dot{\boldsymbol{q}}_d + \boldsymbol{g}(\boldsymbol{q}_d)
\end{equation*}

A diagram of the PD plus feedforward controller is shown in
Fig. \ref{fig:pd-plus-ff}.

\begin{figure}[!t]
  \includegraphics[width=0.5\textwidth]{controller-pd-plus-ff.png}
  \caption{PD Plus Feedforward Control Diagram}
  \label{fig:pd-plus-ff}
\end{figure}

The tuning procedure described in~\cite{kelly2007control} selects gain values for $K_p$
and $K_v$ that guarantee global uniform asymptotic stability, ensuring that the
motion control objective is reached.


\section{Results}
Visualizations of the simulated experiment and main robot configurations are
shown in Figure \ref{fig:configs}. These also include the start and end poses of
the straight line throwing path in task space. The robot begins at the home
configuration before following a joint space planned motion to the start
configuration of the throwing motion.

\begin{figure}[!t]
  \begin{subfigure}{0.24\textwidth}
    \includegraphics[width=\textwidth]{catapult-home-config.png}
    \caption{Home configuration.}
    \label{fig:home-config}
  \end{subfigure}
  \begin{subfigure}{0.24\textwidth}
    \includegraphics[width=\textwidth]{catapult-start-config.png}
    \caption{Configuration at the start of the throw trajectory (straight line in task space).}
    \label{fig:start-config}
  \end{subfigure}
  \begin{subfigure}{0.24\textwidth}
    \includegraphics[width=\textwidth]{catapult-end-config.png}
    \caption{Configuration at the end of the throw trajectory (straight line in task space).}
    \label{fig:end-config}
  \end{subfigure}
  \caption{Main robot configurations during throw motion.}
  \label{fig:configs}
\end{figure}

The full motion sequence was simulated in Gazebo with each controller. The joint
space results are shown in Figure \ref{fig:comp-torque-results} for the computed
torque controller, and Figure \ref{fig:pd-plus-ff-results} for the PD plus
feedforward controller. Most of the motion time, approximately $8$ seconds, is
taken up by moving to the start configuration of the throw motion. The gains for
the computed torque controller were selected as previously described with a
desired critical damping response. Gains for the PD plus feedforward controller
were tuned manually for simplicity.

\begin{figure}[!t]
  \includegraphics[width=0.5\textwidth]{comp-torque-results.png}
  \caption{Desired and actual joint trajectories for throwing motion using a computed torque controller.}
  \label{fig:comp-torque-results}
\end{figure}

\begin{figure}[!t]
  \includegraphics[width=0.5\textwidth]{pd-plus-ff-results.png}
  \caption{Desired and actual joint trajectories for throwing motion using a PD plus feedforward controller.}
  \label{fig:pd-plus-ff-results}
\end{figure}

The results show slight differences in performance for the joint position and
velocity errors. The feedforward controller had lower errors during the motion
up to the start of the throw motion. However smaller errors were observed with
the computed torque controller during the final throw motion. More specifically,
the computed torque controller had less than $1.1$ deg of positional error and
less than $11$ deg/s of angular velocity error.

\section{Challenges and Learning Outcomes}
There were several unexpected challenges during the work of this project. ROS2
was generally found to be more difficult to work with than expected. Although
existing code was leveraged where possible, there were still a few unexpected
issues for simple tasks. The biggest issue was how much time it took to attach
the gripper to the robot. This involved working with many ROS2 packages and URDF
files, and understanding how they work together. In addition, there were
specific control related tags that took a lot of time to understand. The final
ROS related issue appeared when trying to subscribe to the trajectory action
response message in order to read the desired and actual joint values. It took a
reasonable effort to determine the correct message type to subscribe to, as it
was not documented. It required showing message types hidden by default and
searching through generated message files to find the correct message class.

As previously mentioned, it was challenging to reduce the complexity of the
dynamics terms in order to transfer them from Matlab output to C++ code. Setting
some of relative center of mass dimensions to zero helped, as well as neglecting
the coriolis term. The mass matrix was still complex but with some careful find
and replace operations it didn't take too long to put inside the
code. Additionally, it was unexpected that the assumption of only point masses
(no inertia tensor) would cause the computed torque controller to not
generate torque for joint 6. Using a separate PD controller for this joint
addressed the issue, but it did add some complexity.

\section{Conclusion}
This work covered the design and most of the implementation for planning and
controlling the motion of a 6 degree of freedom UR5 robot arm to throw an object
to a target location. The performance of two controllers, computed torque and PD
plus feedforward, were evaluated on tracking the generated trajectories, and it
was found that the computed torque controller performed the best during the
thrown motion.

\section{Future Work}
Future work includes making the robot pick up an object and actually throwing it
toward a target cup. This would likely require further tuning of the controller
for best performance. An additional performance metrics would also include how
close the object is to landing in the target cup, as well as the error in the
projectile trajectory over time.


\bibliographystyle{IEEEtran}
\bibliography{refs}

\end{document}

%%% Local Variables:
%%% mode: latex
%%% TeX-master: t
%%% End:
