\documentclass[conference]{IEEEtran}
\IEEEoverridecommandlockouts
% The preceding line is only needed to identify funding in the first footnote. If that is unneeded, please comment it out.
%Template version as of 6/27/2024

\usepackage{cite}
\usepackage{amsmath,amssymb,amsfonts}

\usepackage{graphicx}
\usepackage{textcomp}
\usepackage{xcolor}
\def\BibTeX{{\rm B\kern-.05em{\sc i\kern-.025em b}\kern-.08em
    T\kern-.1667em\lower.7ex\hbox{E}\kern-.125emX}}


\usepackage{subcaption}

% algorithm psuedocode
\usepackage{algorithm} 
\usepackage{algpseudocode} 


\begin{document}

\title{RBE 550 Catapult Progress Report}

\author{\IEEEauthorblockN{Gregory Kyle Jones}
\IEEEauthorblockA{gkjones@wpi.edu}
}

\maketitle

\section{Introduction}
This report covers the progress of the simulated robot arm catapult
project. There were some changes to the team in which Gregory Jones decided to
continue with the project independently. The goal was to design multiple
controllers for a robot arm to act like a catapult in order to evaluate which
performs best. The robot was simulated to pick up an object and follow a target
motion to throw the object into a target cup some particular distance away. This
project was originally inspired by a separate team doing the same thing with
hardware, however this work was done independently.

\section{Methods}

\subsection{Tools}
Several tools were used to simulate the robot and its environment, plan the
motion, and produce results. Gazebo was used for simulation, MoveIt was used for
motion planning, ROS2 was used for robot communication and control, and Rviz was
used for visual debugging. For simulation, a URDF model of a UR5 robot arm was
used, however a gripper had to be attached and MoveIt configured
accordingly. This simulation setup was used as part of the design process such
that simulation in Matlab was not necessary.

Docker was also used to containerize all of these tools, which helped track
package dependencies. Docker proved to be particularly helpful for making quick
changes to the version of the operating system and software tools. Finally, all
code changes were tracked in a github repository.

\subsection{Projectile Trajectory Planning}
For simplicity the projectile trajectory calculation was represented as a two
dimensional problem in which the motion occurs in an $x-y$ plane. The inputs to
this problem include the release position $(x_0, y_0)$, the final position of
the target cup $(x_f, y_f)$, the release angle $\theta$, and gravity $g$. Using
standard projectile motion equations from~\cite{william25} and rearranging to solve
for the release velocity $v$ gives the equation
\begin{equation*}
  v = \sqrt{ \left(\frac{g(x_f-x_0)^2}{2 cos^2\theta ((x_f-x_0) tan \theta - y_f + y_0)} \right ) }
\end{equation*}

\subsection{Motion Planning}
Several trajectories were planned for the motion of the arm. First for the
motion to pick up the projectile object, and then for the throwing motion. If
the trajectory planning for the throwing motion was done in joint space, then
the end effector would likely be rotating close to the point of releasing the
projectile object. Any small error in the robot timing to release the object
would cause a large error in reaching the target.

To reduce this error amplification, the throwing motion instead used a straight
line end effector path/trajectory in task space, as well as a constant task
space target velocity for some duration leading up to the release point. The
timestamps of each sample point in this linear trajectory were modified based on
the linear segments with parabolic blends (LSPB) method
\cite{mark2020control}. With the distance of the end effector along this linear
path denoted as $s(t)$ where $t$ is time. There is a constant acceleration from
$t_0$ to $t_b$, then a constant velocity from $t_b$ to $t_f-t_b$, and finally a
constant decceleration from $t_f-t_b$ to $t_f$. The desired motion is described
with the following equations.
\begin{equation*}
  s(t) =
  \begin{cases}
    s_0+\frac{\alpha}{2}t^2, & 0 \leq t < t_b \\
    \frac{s_f+s_0-Vt_f}{2} + Vt & t_b \leq t < t_f-t_b \\
    s_f-\alpha {t_f}^2+\alpha t_f t-\frac{\alpha}{2} t^2 & t_f-t_b \leq t_b \leq t_f
  \end{cases}
\end{equation*}

\begin{equation*}
  \dot{s}(t) =
  \begin{cases}
    \alpha t \\
    V \\
    \alpha t_f - \alpha t
  \end{cases}
\end{equation*}

\begin{equation*}
  \ddot{s}(t) =
  \begin{cases}
    \alpha \\
    0 \\
    -\alpha
  \end{cases}
\end{equation*}

\begin{equation*}
  \alpha = \frac{V}{t_b}
\end{equation*}

\subsection{Dynamic Model}
The dynamic model of the selected 6 DoF robot arm (Universal Robots UR5) was
derived analytically using Matlab and Lagrange's method. Although less accurate
than Newton's equations of motion due to using only point masses, Lagrange's
method is easier to derive. Lagrange's method applied to an $n$ DoF manipulator
generates the equations of motion using
\begin{equation*}
  \frac{d}{dt} \left[ \frac{\partial \mathcal{L}(\boldsymbol{q}, \dot{\boldsymbol{q}})}{\partial \dot{\boldsymbol{q}}} \right] - \frac{\partial \mathcal{L}(\boldsymbol{q}, \dot{\boldsymbol{q}})}{\partial \dot{\boldsymbol{q}}} = \boldsymbol{\tau}
\end{equation*}

where $\boldsymbol{q}=[q_1, q_2, ..., q_n]^T$ is the generalized joint
coordinate vector, $\boldsymbol{\tau}$ is the generalized joint force vector,
and $\mathcal{L}(\boldsymbol{q}, \dot{\boldsymbol{q}})$ is the Lagrangian. The
Lagrangian is given by
\begin{equation*}
  \mathcal{L}(\boldsymbol{q}, \dot{\boldsymbol{q}}) = \mathcal{K}(\boldsymbol{q}, \dot{\boldsymbol{q}}) - \mathcal{U}(\boldsymbol{q})
\end{equation*}

where $\mathcal{K}$ is the total kinetic energy, and $\mathcal{U}$ is the total
potential energy due to conservative forces such as gravity and energy stored in
compressed springs.

The final equations of motion can then be represented in compact form as
\begin{equation*}
  M(\boldsymbol{q})\ddot{\boldsymbol{q}} + C(\boldsymbol{q}, \dot{\boldsymbol{q}})\dot{\boldsymbol{q}} + \boldsymbol{g}(\boldsymbol{q}) = \boldsymbol{\tau}
\end{equation*}

where $M(\boldsymbol{q})\ddot{\boldsymbol{q}}$ is the inertia matrix,
$C(\boldsymbol{q}, \dot{\boldsymbol{q}})$ is the coriolis matrix, and
$\boldsymbol{g}(\boldsymbol{q})$ is the gravity vector.

\subsection{Controllers}
Only controllers that satisfy the motion control objective were selected for
implementation. That is they produce $\boldsymbol{\tau}$ such that the joint
positions $\boldsymbol{q}$ follow the time changing desired joint positions
$\boldsymbol{q_d}$ accurately. The selected controllers include computed torque
and PD plus feedforward~\cite{kelly2007control}. Both of these controllers take into account the full
dynamic model.

\subsubsection{Computed Torque}
The controller for computed torque uses the robot dynamics in the feedback loop
and is given as
\begin{equation*}
  \boldsymbol{\tau} = M(\boldsymbol{q}) \left [ \ddot{\boldsymbol{q}}_d + K_v \dot{\tilde{\boldsymbol{q}}} + K_p \tilde{\boldsymbol{q}} \right ] + C(\boldsymbol{q}, \dot{\boldsymbol{q}})\dot{\boldsymbol{q}} + \boldsymbol{g}(\boldsymbol{q})
\end{equation*}

where $K_v$ and $K_p$ are symmetric positive definite design matrices and
$\tilde{\boldsymbol{q}} = \boldsymbol{q}_d - \boldsymbol{q}$ is the position
error. These design matrices can be selected to be diagonl, which cause the
dynamics of errors of the closed loop system to become decoupled, such that the
behavior of each joint position error is governed by a second order linear
differential equation. These matrices can then be selected as
\begin{align*}
  K_p &= diag\{\omega_1^2, ... , \omega_n^2\} \\
  K_v &= 2 K_p
\end{align*}

which causes each joint to respond as a critically damped linear system with
bandwith $\omega_i$. Bandwidth $\omega_i$ defines the allowable range for joint
velocity $i$ before more than a $90$ degree phase lag occurs, as well as the
decay rate of the error for joint $i$.

\subsubsection{PD Plus Feedforward}
The controller for PD plus feedforward uses the dynamics in a feedforward manner
by using the position and velocity of the desired trajectory. It also includes a
PD feedback controller. The complete controller is given by
\begin{equation*}
  \boldsymbol{\tau} = K_p \tilde{\boldsymbol{q}} + K_v \dot{\tilde{\boldsymbol{q}}} + M(\boldsymbol{q}_d)\ddot{\boldsymbol{q}}_d +  C(\boldsymbol{q}_d, \dot{\boldsymbol{q}}_d)\dot{\boldsymbol{q}}_d + \boldsymbol{g}(\boldsymbol{q}_d)
\end{equation*}

The tuning procedure described in~\cite{kelly2007control} selects gain values for $K_p$
and $K_v$ that guarantee global uniform asymptotic stability, ensuring that the
motion control objective is reached.

\section{Results}

The following results will be produced for each controller with at least two
different projectile target distances for performance evaluation:

\begin{itemize}
\item A demonstration video of the end-to-end sequence for each controller will
  be captured. This will visually show whether the projectile reaches the target or not.
\item A 2D plot of the desired, actual, and error joint trajectories over time.
\item A 2D plot of the desired, actual, and error task space trajectories over time.
\end{itemize}

The last two items will show how well the controllers track the desired trajectories.

\section{Conclusion}
An evaluation of the results for each controller will be performed covering the
performance of trajectory tracking as well as how close the projectile comes to
reaching the projectile target (a cup) for various distance. These results will
be compared to determine which controller is best.

\bibliographystyle{IEEEtran}
\bibliography{refs}

\end{document}

%%% Local Variables:
%%% mode: latex
%%% TeX-master: t
%%% End:
